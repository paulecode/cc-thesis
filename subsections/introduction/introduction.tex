\documentclass[../../sections/introduction.tex]{subfiles}
\begin{document}

    Various composers shaped classical music over the course of time.
    Each composer possesses unique characteristics, resulting in a wide
    range of facettes for classical music. The work of some composers
    sometimes inspired later composers, which leads to some pieces
    containing similarities, despite being composed by a different
    composer in a different era. The same composers would also not only
    compose a single musical form, but usually boasted a wide variety of
    musical forms in their repertoire, like slower and more somber
    Nocturnes, or more precise and fast-paced Etudes. For many trained
    musicians, it is an easy task to intuitively classify and associate a
    given piece to a composer and musical form. A lot of research has
    been done in the field of music information retrieval (MIR) on
    extracting information like genre, mood, artist from songs, albeit
    often on more modern music. Since classical music datasets are more
    sparse, there is less research than on more popular music, however,
    there is a dataset of recordings from various piano competitions,
    totaling over 200 hours, represented by .wav files and .midi files
    simultaneously. In this work, I want to explore and answer the
    following research questions:

    \begin{itemize}

      \item \textbf{RQ1:} What is the feasability of symbolic data in
        MIDI for MIR compared to audio source data in .wav files?

      \item \textbf{RQ2:} How can a machine learning model be trained and
        optimized?

        \begin{itemize}

          \item \textbf{RQ2.1:} What features are the most relevant?

          \item \textbf{RQ2.2:} Which algorithms perform the best?
          
        \end{itemize}
      
    \end{itemize}

\end{document}
